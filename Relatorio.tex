\documentclass[10pt,a4paper]{article}

\usepackage{indentfirst}
\usepackage{amsthm,amsfonts,amsmath,amssymb}
\usepackage[brazilian]{babel}
\usepackage[T1]{fontenc}
\usepackage[utf8]{inputenc}
\usepackage{setspace}
\usepackage[usenames,dvipsnames]{xcolor} 
\usepackage{pgf,tikz}
\usepackage{float}
\usepackage{graphicx}
\usepackage{subfigure}
\usepackage{wrapfig}
\usepackage{multirow}
\usepackage{xcolor,colortbl}
\usepackage{changepage}
\usepackage{geometry}
\usepackage[pdftex]{hyperref}
\usepackage{listings}
\usepackage[normalem]{ulem}
\usepackage{enumitem}
\usepackage{booktabs}
\usepackage{multirow,array,varwidth}
\usepackage{tabularx}
\usepackage{makeidx}
\usepackage[nottoc]{tocbibind}
\usepackage{caption}
\usepackage{etoolbox}
\usepackage{longtable}
\usepackage{calc}
\geometry{a4paper,inner=2.0cm,outer=2.0cm,top=2.0cm,bottom=2.0cm}
\setenumerate[1]{label=\thesubsection.\arabic*.}
\setenumerate[2]{label*=\arabic*.}

\setlength{\tabcolsep}{6pt}

\newcommand*\NewPage{\newpage\null\thispagestyle{empty}\newpage}
\newcommand{\Barra}{\ensuremath{\backslash}}

\newcommand\Data[3]{\ensuremath{#1\backslash #2\backslash #3}}

\newcounter{magicrownumbers}
\newcommand\rownumber{\stepcounter{magicrownumbers}\arabic{magicrownumbers}}

\setlength{\columnsep}{1cm}
\addto\captionsbrazilian{% Replace "english" with the language you use
  \renewcommand{\contentsname}%
    {Tabela de Conteúdo}%
}

\begin{document}

\thispagestyle{empty}
\begin{center}
	UNIVERSIDADE DE SÃO PAULO – USP
	
	INSTITUTO DE CIÊNCIAS MATEMÁTICAS E DE COMPUTAÇÃO
	
	DEPARTAMENTO DE SISTEMAS DE COMPUTAÇÃO
	
	\vspace{7cm}
	
	\Large{\textbf{Relatório Sistema Aeronave}}\\
	
	\vspace{6cm}
	
	Adams Vietro Codignotto da Silva - $6791943$ \\ 
	Sabrina Faceroli Tridico - $9066452$\\
	Lucas Fronza - $8124184$\\
	
		
	
	\vspace{6cm}
	
	São Carlos
	
	2017
\end{center}

\NewPage
\pagenumbering{arabic}

\tableofcontents

\newpage

\section{Introdução}
Neste projeto foi utilizado a linguagem C/C++, desenvolvido e testado apenas em ambientes Linux e Mac, e seu funcionamento em Windows não é garantido. Foram utilizadas bibliotecas padrões nativas do gcc, então não é necessário de nenhum pacote adicional.
\section{Socket}
Para a implementação do socket, foram utilizadas principalmente as bibliotecas \textit{sys/socket.h} e \textit{netdb.h}. Foi escolhido uma abordagem de cliente/servidor utilizando conexão UDP com chamada bloqueante, pois foi esta que mais tivemos documentações disponíveis online. Outras funções exclusivas do sistema Linux foram utilizadas, como \textit{getaddrinfo} e \textit{freeaddrinfo}.\\
A estrutura do cliente e do servidor são parecidas, ambos construtores recebem um endereço e uma porta para comunicar, e recebem as mesmas propriedades de conexão (como protocolo e tipo de socket), porém o servidor é associado a um único endereço, criando uma chamada bloqueante.\\
As funções send e recv foram utilizadas para enviar e receber mensagens pelo endereço/porta definidos nos construtores.
\section{Sensores}
%------COMENTÁRIO COM %
%Só digitar normal, \textbf{text} é negrito, \textit{text} é itálico
%Pra pular a linha é só colocar \\ no final do parágrafo

%digita aqui a parte do sensor
\section{Manual de Usuário}
%digita aqui a parte do manual do usuário

%\section{Conclusão} -- precisa?
\end{document}
